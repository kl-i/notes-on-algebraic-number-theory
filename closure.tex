\documentclass[./main.tex]{subfiles}
\begin{document}
  
\begin{rmk}[Results of this section]~
  
  \begin{itemize}
    \item Basic Properties of Trace and Norm
    \item Trace Characterisation of Finite Separable Extensions
    \item (Main) Integral Closure of Dedekind Domain in Finite 
    Extension is Dedekind. 
  \end{itemize}

  I reluctantly wrote about the trace and norm
  since it looks like ANT cannot theoretically do without them,  
  but I still do not have geometric intuition for them. 
\end{rmk}

\begin{prop}[Trace Characterisation of Finite Separable Extensions]
  
  Let $K \to L$ be a finite extension.
  Then it is separable if and only if 
  the bilinear form $L \times L \to K, \al,\be \to \tr (\al \be)$
  is non-degenerate. 
\end{prop}
\begin{proof}(From Janusz) Don't want to think about trivial case of $K = L$,
  so we assume $K \to L$ is a non-trivial extension.

  $(\implies)$
  We use a clever $K$-basis of $L$ to link 
  the determinant of the trace form and separability of $L$ over $K$.
  The clever basis is : 
  let $L = K \oplus K\theta \oplus \cdots \oplus K\theta^{[L:K]-1}$
  by the primitive element theorem so that 
  given a finite Galois extension $K \to \Om$ 
  with $K\ALG(L,\Om) = \set{\si_1,\dots,\si_{[L:K]}} \neq \nothing$,
  it suffices 
  \[
    \det \sqbrkt{\tr_{L/K}(\theta^{i-1}\theta^{j-1})}
    = \prod_{i < j} \brkt{\si_i(\theta) - \si_j(\theta)}^2
  \]
  since the latter is non-zero by separability of $L$ over $K$.

  So let $K \to \Om$ be a finite Galois extension
  with $K\ALG(L,\Om) = \set{\si_1,\dots,\si_{[L:K]}} \neq \nothing$.
  Then we have the algebraic trick called the Vandermonde matrix : 
  \[
    \det \begin{bmatrix}
      1 & &                         1                       \\
      \si_1(\theta) & &             \si_{[L:K]}(\theta)           \\
      \vdots & \cdots &             \vdots       \\
      \si_1(\theta)^{[L:K] - 1} & & \si_{[L:K]}(\theta)^{[L:K] - 1}
    \end{bmatrix}
    = \prod_{i < j} \brkt{\si_i(\theta) - \si_j(\theta)}
  \]
  So it suffices that $\sqbrkt{\tr_{L/K}(\theta^{i-1}\theta^{j-1})} = VV^t$.
  We have \[
    [VV^t]_{i,j} = \dsum{k}{} \si_k(\theta^{i - 1})\si_k(\theta^{j-1})
    = \dsum{k}{} \si_k(\theta^{i - 1}\theta^{j-1})
  \]
  So it suffices that for any $\theta^l$, 
  the trace $\tr_{L/K}(\theta^l) = \dsum{k}{} \si_k(\theta^l)$ in $\Om$.
  To see this for $l = 1$, 
  note that $\ch(\theta,K) = \min(\theta,K)$ since 
  the former has $\theta$ as a root and has the same degree as the latter.
  To get arbitrary $l$, 
  note that $\ch(\theta,K) = \min(\theta,K)$ separable implies 
  $[\theta]$ is diagonalisable in $\Om$ with eigenvalues 
  $\si_1(\theta),\dots,\si_{[L:K]}(\theta)$.
  It follows that $[\theta^l]$ is diagonalisable in $\Om$ as well,
  with eigenvalues $\si_1(\theta^l),\dots,\si_{[L:K]}(\theta^l)$.

  $(\limplies)$
  Suppose $L$ is inseparable over $K$,
  so we have characteristic of $K$ being some prime $p > 0$,
  $K \to L_S \to L$ where $K \to L_S$ is separable and 
  $L_S \to L$ is purely inseparable with degree $p^N$ for some $N > 0$.
  We need to give an $x \in L$ such that 
  for all $y \in L$, $\tr_{L/K}(xy) = 0$ but $x \neq 0$.
  By assumption, we have an $x \in L \minus L_S$.
  Then for $y \in L$, $\tr_{L/K}(xy) = \tr_{L_S/K}\brkt{\tr_{L/L_S}(xy)}$
  so it suffices $\tr_{L/L_S}(xy) = 0$.
  Well, if $xy \in L_S$, 
  then $\tr_{L/L_S}(xy) = p^N xy = 0$.
  If $xy \notin L_S$, 
  then there exists $n > 0$ such that $\min(xy,L_S) = T^{p^n} - a$
  for some $a \in L_S$.
  By working in an extension $L \to \Om$ where $\ch(xy,L_S)$ splits,
  one sees that $\ch(xy,L_S)(T) = (T - xy)^{p^N}$ in $\Om$,
  and so $\ch(xy,L_S)(T) = (T^{p^n} - a)^{p^{N-n}}$,
  which implies $\tr_{L/L_S}(xy) = 0$ again. 
  
\end{proof}

\begin{prop}[Integral Closure of Dedekind Domain in Finite Extension]
  
  Let $A$ be a Dedekind domain, 
  $K$ its field of fractions,
  $K \to L$ a finite extension of fields,
  $B$ the integral closure of $A$ in $L$.
  Then $B$ is a Dedekind domain. 
\end{prop}
\begin{proof}(Milne, Janusz combined)
  
  \textit{(Integrally closed)} Transitivity of being integral over a base ring.

  \textit{(Dimension 1)}
  Let $q \in \spec B$ be non-generic and $\pi : \spec B \to \spec A$
  the adjunct of $A \to B$. 
  Let $p = \pi(q)$.
  Then $0 \to A/I(p) \to B/I(q)$ is an integral extension of integral domains.
  \begin{lem}
    For $A \subs B$ ID where $B$ is integral over $A$,
    $B$ is a field if and only if $A$ is.
    \begin{proof1}
      Atiyah. The argument is elementary. 
    \end{proof1}
  \end{lem}
  So it suffices to prove $p$ is not the generic point of $\spec A$.
  Well, there exists $f \in B\minus 0$ that vanishes at $q$.
  Since $B$ integral over $A$, $f^n + a_1 f^{n - 1} + \cdots + a_0 = 0$
  for some $a_k \in A$.
  Let $n$ be minimal. Then $0 \neq a_0 \in I(p)$.

  \textit{(Noetherian)}
  We have the decomposition $K \to L_{sep} \to L$ where 
  $K \to L_{sep}$ is separable and $L_{sep} \to L$ is purely inseparable. 
  We hence also have a decomposition $A \to A_{sep} \to B$
  where $A_{sep}$ be the integral closure of $A$ in $L_{sep}$
  and it follows that $B$ is the integral closure of $A_{sep}$ in $L$.
  It thus suffices that $B$ is Noetherian over $A_{sep}$ and 
  $A_{sep}$ is Noetherian over $A$.

  We first prove $A_{sep}$ Noetherian over $A$ by 
  proving a more detailed lemma
  which gives us extra data for the example of algebraic integers. 
  \begin{lem}
    Let $A$ be an integrally closed domain, 
    $K \to L$ a finite separable field extension where 
    $K$ is the fraction field of $A$ and 
    $B$ the integral closure of $A$ in $L$.
    Then there exists free sub-$A$-modules $M, M_1 \subs L$ with rank $[L:K]$
    such that $M \subs B \subs M_1$.
    
    In particular, $A$ Noetherian implies $B$ is Noetherian. 
    Also, $A$ PID implies $B$ is also a free $A$-module with rank $[L:K]$.

    \begin{proof1}
      Let $L = K\be_1 \oplus \cdots \oplus K\be_{[L:K]}$.
      For any $x \in L$, there exists $a \in A\minus 0$ such that $ax \in B$. 
      So we can WLOG assume $\be_1,\dots,\be_{[L:K]} \in B$.
      Set $M := A\be_1 + \cdots + A\be_{[L:K]}$.
      Since $K \to L$ is finite separable, 
      non-degeneracy of the trace form gives another $K$-basis 
      $\be_1',\dots,\be_{[L:K]}'$ of $L$ that is 
      dual to the previous basis with respect to the trace form,
      i.e. $\tr_{L/K}(\be_i \be'_j) = \de_{i,j}$.
      Let $M_1 := A\be'_1 + \cdots + A\be'_{[L:K]}$.
      It remains to show $B \subs M_1$.
      Let $b \in B$. Then $b = \dsum{i}{} \la_i \be'_i$ for some $\la_i \in K$.
      Then $
        \la_j = \tr_{L/K}(b \be_j) \in \tr_{L/K} B \subs A
      $ by the two bases being dual w.r.t. the trace form. 
    \end{proof1}
  \end{lem}
  So we have $A_{sep}$ is a Dedekind domain. 

  Now we prove $B$ is Noetherian over $A_{sep}$.
  According to Janusz, this is quite hard so 
  we directly show $B$ is Dedekind by showing 
  all non-zero functions have finite vanishing and 
  all stalks are DVRs. TODO.

\end{proof}

\end{document}