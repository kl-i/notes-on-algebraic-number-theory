\documentclass[./main.tex]{subfiles}
\begin{document}
  
\begin{dfn}[Noetherian, Integral, Krull Dimension]

  Let $X \in \AFF$ be an affine scheme. 
  \begin{itemize}
    \item $X$ is called \emph{Noetherian} when 
    any of the following equivalent conditions are met : 
    \begin{enumerate}
      \item $\OO_X(X)$ is Noetherian.
      \item for all opens $U \subs X$, $\OO_X(U)$ Noetherian.
      \item there exists an open cover $\UU$ of $X$ such that 
      for all $U \in \UU$, $\OO_X(U)$ Noetherian. 
    \end{enumerate}
    \item $X$ is called \emph{integral} when 
    any of the following equivalent conditions are met : 
    \begin{enumerate}
      \item $\OO_X(X)$ is an integral domain.
      \item the underlying topological space of $X$ is irreducible and 
      $X$ is reduced. 
    \end{enumerate}
    \item $\dim X := $ Krull dimension of $\OO_X(X)$. 
    $X$ is called a \emph{curve} when $\dim X = 1$.
    \item Call $X$ \emph{local}\footnote{
      I made this up. 
    } when 
    any of the following equivalent definitions are met : 
    \begin{enumerate}
      \item $X$ has a unique closed point.
      \item $\OO_X(X)$ is a local ring. 
    \end{enumerate}
    \item Let $X$ be integral. 
    Then $X$ is \emph{integrally closed} when 
    any of the following equivalent definitions are met : 
    \begin{enumerate}
      \item $\OO_X(X)$ is integrally closed.
      \item For all points $x \in X$, $[\OO_X]_x$ is integrally closed.
      \item For all closed points $x \in X$, $[\OO_X]_x$ is integrally closed.
    \end{enumerate}
  \end{itemize}
\end{dfn}

\begin{rmk}
  Still no geometric intuition for integrally closed. 
\end{rmk}

\begin{prop}[Characterisation of DVRs]

  Let $X \in \AFF$ be a local Noetherian integral curve. 
  Let $p$ be its unique closed point and $\ka(p)$ its residue field. 
  Let $K := [\OO_X]_{p_X}$ where $p_X$ is the unique generic point of $X$,
  i.e. $K = \mathrm{Frac}\, \OO_X(X)$.
  Then TFAE : 
  \begin{enumerate}
    \item There exists a valuation $v : K^\times \to \Z$ with 
    $\OO_X(X)$ as its valuation ring. 
    \item $\OO_X(X)$ is integrally closed. 
    \item $\ker\ev_{p}$ is principal. 
    \item $\dim_{\ka(p)} T_{p}^* X = 1$,
    i.e. $X$ is smooth. 
    \item The only ideals of $\OO_X(X)$ are $(0)$ and powers of $I(p)$.
    That is to say,
    the only closed subschemes of $X$ are $X$ and 
    the infinitesimal neighbourhoods of $p$.
    \item There exists $f \in \OO_X(X)$ such that 
    all non-zero ideals of $\OO_X(X)$ are of the form 
    $(f^k)$ for some $k \in \N$.
  \end{enumerate}
  $\OO_X(X)$ is called a \emph{discrete valuation ring} when 
  any (and thus all) of the above are satisfied. 

\end{prop}

\end{document}